%%%%%%%%%%%%%%%%%%%%%%%%%%%%%%%%%%%%%%%%%
% "ModernCV" CV and Cover Letter
% LaTeX Template
% Version 1.1 (9/12/12)
%
% This template has been downloaded from:
% http://www.LaTeXTemplates.com
%
% Original author:
% Xavier Danaux (xdanaux@gmail.com)
%
% License:
% CC BY-NC-SA 3.0 (http://creativecommons.org/licenses/by-nc-sa/3.0/)
%
% Important note:
% This template requires the moderncv.cls and .sty files to be in the same
% directory as this .tex file. These files provide the resume style and themes
% used for structuring the document.
%
%%%%%%%%%%%%%%%%%%%%%%%%%%%%%%%%%%%%%%%%%

%----------------------------------------------------------------------------------------
%	PACKAGES AND OTHER DOCUMENT CONFIGURATIONS
%----------------------------------------------------------------------------------------

% Font sizes: 10, 11, or 12;
% paper sizes: a4paper, letterpaper, a5paper,legalpaper,
%			   executivepaper or landscape;
% font families: sans or roman
\documentclass[11pt,letterpaper,sans]{moderncv}

\moderncvstyle{classic}
\moderncvcolor{blue}

\usepackage{anyfontsize}
\usepackage[scale=0.85]{geometry} % Reduce document margins
% \setlength{\hintscolumnwidth}{3cm} % Uncomment to change the width of the dates column
% \setlength{\makecvtitlenamewidth}{10cm} % For the 'classic' style, uncomment to adjust the width of the space allocated to your name

\usepackage{CJKutf8}

%----------------------------------------------------------------------------------------
%	NAME AND CONTACT INFORMATION SECTION
%----------------------------------------------------------------------------------------

\firstname{Taylor} % Your first name
\familyname{McKinney} % Your last name

% All information in this block is optional, comment out any lines you don't need
% \title{}
\address{Tokyo, Japan}{}
\mobile{(+81)-80-7265-5051}

% The first argument is the url for the clickable link,
% the second argument is the url displayed in the template
% This allows special characters to be displayed such as the tilde in this example
\email{taylorsmck@gmail.com}
% \homepage{url}{text}
% \extrainfo{additional information}

% --- begin doc ---

\begin {document}
\begin {CJK}{UTF8}{min}
\makecvtitle % Print the CV title

% --- work experience ---
\section{Experience}
\cventry
{2023--Present}
{Revcomm}
{Senior Software Engineer - Full Stack}
{Tokyo}{Japan}{}

\cvlistitem {
Built and maintained an application to manage the authentication and management of users,
including clients, their users, and of Revcomm's internal users.
This includes several APIs for managing user data,
and multiple front-end applications.
}

\cventry
{2022--2023}
{SmartNews}
{Senior Software Engineer - Front-End}
{Tokyo}{Japan}{}

\cvlistitem{
Developed a Webview component for a mobile app that could be used to display news stories and
a variety of additional content that could not be handled by the native components.
This was my introduction to mobile development, and I learned a lot about building web applications
designed to run inside mobile applications. This project relied on vanilla JavaScript, but used
Rollup and a lot of bash scripts for deployment.
}

\cvlistitem{
Worked on a WYSIWYG editor that non-engineering employees could use to create a variety of content,
such as pages displaying news related to a specific event, or new visual data components. This
project used React itself, but supported adding components made by other teams which could be
vanilla web components, React components, or Svelte components.
}

\cvlistitem{
Developed internal web portal for Advertising clients to manage their ads, ad groups, and campaigns.
This brought the company up to modern standards for advertisment management. It also served as a
testbed for some of the latest front-end libraries, such as Vite, pnpm, and ReactQuery.
}

\cventry
{2020--2022}
{AXA}
{Senior Software Engineer - Front-End}
{Tokyo}{Japan}{}

\cvlistitem{
Developed tool to send customers automated updates on their claims via SMS. This service ran on
AWS Lambda functions, and used S3 for storage some simple details, such as shorted URLs which could
fit in an SMS message.
}

\cvlistitem{
Developed an internal API authentication tool that allowed teams to secure any APIs we created in
a one-size-fits-all manner. This freed up API teams from needing to design their own security
measures. This tool made use of some AWS Lambda functions and some neat API Gateway configuration.
}

\cvlistitem{
Created a web portal for employees to manage sales representive contracts, as well as project the
expected payment for contracted sales reps out to six months. This project used Create React App for
the front-end, and Java for the API, and a z/OS mainframe for the back-end.
}

\cvlistitem{
Designed the architecture for a multitude of applications and tools as requested by teams
throughout the company.
}

\cvlistitem{
Became a leader for the JavaScript Community of Practices.
Hosted weekly meetups for everyone to discuss the latest changes in JavaScript, discuss new tools,
and have the occasional friendly bike-shed about our company-wide practices. We also held small quizes
to demonstrate useful new features, and all worked together to create the standards for front-end
development at the company. Additionally, I led training sessions on topics including Next.js,
internationalization, accessibility, and Redux.
}

\cventry
{2019--2020}
{Netsmile}
{Systems Software Engineer - Full-Stack}
{Tokyo}{Japan}{}

\cvlistitem{
Developed an API service layer and web portal for clients. This application allowed users to upload
photos and documents, forward those files along with client configuration to an AI pipeline for OCR
and image recognition tasks, then report the results to the user. This application used Next.js for
the front-end, Knex.js and PostgreSQL for metadata and client config, S3 for storing copies of
output files, and RabbitMQ to interface with the AI pipeline.
}

\clearpage
\cventry
{2013--2019}
{Bazaarvoice}
{Senior Software Engineer - Front-End}
{Austin}{USA}{}

\cvlistitem{
Developed 3rd-party application to display ratings and reviews for retail clients.
With over 400 million unique visitors per month, work on this application was high pressure, and
it's where I first worked in front-end development.
The application itself was built using a combination of Backbone and jQuery, and had strict
requirements for performance and user experience.
}

\cvlistitem{
Early work on this project involved improving accessibility support. This involved learning the
WCAG 2.0 standard and how to meet it, how to use and test various software for supporting impaired
users, and learning a mental framework for how to approach accessibility early and effectively.
}

\cvlistitem{
Completely overhauled the build system for the application to scale better with growing use. As our
client base grew, the build system needed to handle an order of magnitude more messages. I was able
to completely retool the build system to use an solution. This solution took the form
of a docker image running in an autoscaling group in EC2 that processed messages coming from SQS
and SNS, and started new instances based on the size of the queue. Once this overhaul was done, the
build system could handle any number of builds in under 10 minutes.
}

\cvlistitem{
Managed client satisfaction for all display applications. Developed new company processes to quickly
triage tickets and route them to the appropriate teams. Later developed runbooks for the most
common reaccuring issues, and trained support teams to use the runbooks to solve these issues
immediately, avoiding the need to wait for developers to respond. This gave me plenty of practice
debugging client code on live sites, taught me many ways to approach identifying and solving errors,
and also helped me develop soft skills such as communicating with clients and support staff.
}

\cvlistitem{
Trained and lead international team of engineers to take over several applications, including the
front-end application mentioned above. This involved helping them get familiar with the applicaiton,
teaching them the various processes we had developed to keep the application running smoothly, and
slowly transitioning myself out.
}

% --- skills ---

\section{Skills}

\cvitem{English}{Native Speaker}
\cvitem{Japanese}{JLPT N2 - Business-level proficiency}
\cvitem[11pt]{Languages}{Javascript, Typescript, C++, C\#, Python, Java}
\cvitem[11pt]{Front-End}{Next.js, React.js, Create React App, Redux, Webpack, Node, Express, Hapi, Jest, Vite}
\cvitem[11pt]{Testing}{Jest, Vitest, Mocha, AVA, Node Tap}
\cvitem[11pt]{Back-End}{SpringBoot, Django, Knex.js, Sequelize, MySQL, PostgreSQL}
\cvitem[11pt]{AWS}{SQS, SNS, SES, EC2, ECS, EKS, RDS, Lambda, API Gateway, CloudFront, Route 53, VPC, S3, DynamoDB, ElastiCache, AWS Auto Scaling, CloudFormation, CloudWatch, Certificate Manager, KMS, Secrets Manager}
\cvitem[11pt]{Pipeline}{Jenkins, CircleCI, Github Actions, OpenShift, TravisCI}
\cvitem[11pt]{Assorted}{Serverless, Docker, SonarQube, Terraform, Kubernetes}

\section{Education}

\cventry
{2008-2013}
{Bachelor of Science in Computer Science}
{The University of Texas}
{Austin}{}{}

% --- end document ---
\end {CJK}
\end {document}
